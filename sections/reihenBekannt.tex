% % % % % % % % % % % % % % % % % %
%Bekannte Reihen
% % % % % % % % % % % % % % % % % %	
\begin{tabularx}{\textwidth}{|p{180pt}|p{180pt}|X|}
	\hline
	\rowcolor{Gray}
	\multicolumn{3}{|c|}{\textbf{Bekannte Reihen} \qquad \fb{S.19-21, 477-478, (478)}}\\
	\hline
		\underline{Geometrische:}\vspace{1mm}\newline
		 $s_n=\sum\limits_{k=0}^{n}a_0 \cdot q^{k}=a_0\cdot\dfrac{1-q^n}{1-q}$
		 \newline
		 $s=\sum\limits_{k=0}^{\infty}a_0\cdot q^{k}=\dfrac{a_0}{1-q}$
		 \newline
		 $\sum\limits_{k=0}^{\infty}a_0 \cdot q^{k}=$
		 $\begin{cases}
		 	q = 0 & undef\\
			 |q| < 1 & 1\\
			 q<(-1) & \pm \infty \\
			 q>1 & +\infty 
		 \end{cases}$
		&		 
		\underline{p-Reihe:}\vspace{1mm}\newline 
		$\sum\limits_{n=0}^{\infty}\dfrac{1}{n^p}=$
		$\begin{cases}
			p>1 & konvergent\\
			p\leq1 & divergent
		\end{cases}$
		&
		\underline{Potenz-Reihe:}\vspace{1mm}\newline 
		$\sum\limits_{n=1}^{\infty}\dfrac{x^n}{n^\alpha} \quad\Rightarrow \quad \rho=\lim _{n \rightarrow \infty} \sqrt[n]{n^{\alpha}}=1$ \vspace{1mm}\newline
		Randwerte:\vspace{1mm}\newline 
		$x = 1 \Rightarrow
		\begin{cases}
			\alpha > 1 & konvergiert\\
			\alpha \leq 1 & divergiert\\
		\end{cases}$
		\newline
		$x = (-1) \Rightarrow
		\begin{cases}
		\alpha > 0 & konvergiert\\
		\alpha \leq 0 & divergiert\\
		\end{cases}$
		\\
	\hline
		\underline{Arithmetische:}\vspace{1mm}\newline 
		$s_n=\sum\limits_{k=0}^{n}a_0 +k\cdot d=\frac{n}{2}(a_1+a_n)$
		&
		\underline{Exponentialfunktion:}\vspace{1mm}\newline
		$\sum\limits_{n=0}^{\infty}\dfrac{x^n}{n!}=e^x = 1+x+\frac{x^{2}}{2}+\frac{x^{3}}{6}+\ldots(\rho=\infty)$
		&
		\underline{Binominal-Reihe}\vspace{1mm}\newline 
		$\sum\limits_{n=0}^{\infty}\binom{\alpha}{n}\cdot x^n = (1+x)^\alpha\quad$
		$|\rho| = 1$\newline
		p.m. $\binom{u}{k}= \frac{u!}{(u-k)!k!}$
		\\
	\hline
		\underline{Harmonische:} (divergiert)\vspace{1mm}\newline 
		$s_n= \sum\limits_{k=1}^{n}\dfrac{1}{k}=1+\frac{1}{2}+\frac{1}{3}...$
		&
		\underline{alternierende Harmonische:} (bedingt konvergent)\vspace{1mm}\newline 
		$\sum_{n=1}^{\infty}(-1)^{(n+1)} \frac{1}{n} = \ln (2)$
		&
		Spezialfall (Binominalreihe): $\Rightarrow \alpha=\frac{1}{2}$  \vspace{1mm}\newline
		$(1+x)^{1 / 2}=\sqrt{1+x}=1+\frac{x}{2}-\frac{x^{2}}{8} \pm \cdots (\rho=1)$
		\\
	\hline
\end{tabularx}	

\end{center}
\end{table}	