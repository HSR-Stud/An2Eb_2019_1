\clearpage

\begin{table}[h!]
\section{Reihen}

\begin{center}

% % % % % % % % % % % % % % % % % %
%Grundlegendes
% % % % % % % % % % % % % % % % % %	
\begin{tabularx}{\textwidth}{|p{100pt}|X|}
\hline
\rowcolor{Gray}
\multicolumn{2}{|c|}{\textbf{Grundlegendes}}\\
\hline
	Reihe & 
	Folge $\langle a_n \rangle = a_1,a_2...a_n \qquad$
	Folge $\langle s_1\rangle = a_1$ und $\langle s_2 \rangle = a_1+a_2$\newline
	Eine Reihe ist eine Folge ihrer Partialsummen: \quad $\lim\limits_{n\to\infty} s_n  =  \lim\limits_{n\to\infty}\sum\limits_{k=1}^{n}a_k = \sum\limits_{k=1}^{\infty}a_k = s$\\
\hline
	Konvergenz/Divergenz &
	Konvergiert die unendliche Reihe $\langle s_n\rangle $so besitzt sie die Summe s. $\qquad
	s=\sum\limits_{k=1}^{\infty} a_k$\newline
	Existiert der Grenzwert nicht, so ist die Reihe divergent.\newline
	Wenn man in einer Reihe endlich viele Summanden hinzu/weglässt, so bleibt sie Konvergent oder Divergent. (nicht so bei Folge)\\
\hline
	Vertauschen \newline der Summanden &
	Für \textbf{unendliche Reihen} gilt, dass die einzelnen Summen untereinander \underline{nicht} vertauscht werden können\\
\hline
	Es gilt ausserdem&
	$a=\sum\limits_{k=1}^{\infty} a_k \quad b=\sum\limits_{k=1}^{\infty} b_k$ \ sind konvergente Reihen $\quad a_k \leq b_k\quad \forall n\in \mathbb{N} \qquad \Rightarrow \quad a\leq b$\\
	\hline
\end{tabularx}

% % % % % % % % % % % % % % % % % %
%Konvergenzkriterien
% % % % % % % % % % % % % % % % % %	
\begin{tabularx}{\textwidth}{|p{100pt}|X|}
\hline
\rowcolor{Gray}
\multicolumn{2}{|c|}{\textbf{Konvergenzkriterien} \qquad \fb{S.472-476}}\\
\hline
	Cauchyches Konvergenzkrit.&
	Es existiert ein $\epsilon > 0 \qquad \epsilon \geq s_0=\sum\limits_{k=1}^{n_0}$\qquad
	Nun gilt für alle $m>n>n_0 \qquad \vert \sum\limits_{k=n}^{m}a_k\vert <\epsilon$\newline
	Dann Konvergiert die Reihe, ansonsten divergiert sie. $(|s_m-s_n|< \epsilon)$	 \\
\hline
	Reziprokkrit&
	$ s = \sum\limits_{n=1}^{\infty} \dfrac{1}{n^\alpha} $
	$\begin{cases}
	$konvergent für$ & \alpha > 1\\
	$divergent für $ & \alpha \leq 1
	\end{cases}$\\
\hline
	Trivialkriterium \newline
	\fb{S. 473  (7.2.1.2) }  & 
	$\sum\limits_{n=1}^{\infty} a_n \quad
	\begin{cases}
	\lim\limits_{n\to\infty}a_n \neq 0 & \text{divergent}\\
	\lim\limits_{n\to\infty}a_n = 0 & \text{konvergent oder divergent} \Rightarrow \text{weitere Tests notwendig!} \\
	\end{cases}$\\
\hline
	Majorantenkrit.  \newline
	\fb{S. 479  (7.2.5.1)}&
	Ist die Reihe $ \sum\limits_{n=1}^{\infty} c_n $ konvergent, so konvergiert auch die Reihe $ \sum\limits_{n=1}^{\infty} |a_n|$ und somit auch
	$\sum\limits_{n=1}^{\infty} a_n$ für $|a_n| \leq c_n$ (absolut).
	\newline Dies gilt auch für $|a_n| \leq c_n$ erst ab einer Stelle $n_0 \in \mathbb{N}$.
	\\
\hline
	Minorantenkrit.  &
	Ist die Reihe $ \sum\limits_{n=1}^{\infty} d_n $ gegen $+\infty$ divergent, so gilt dies auch für die Reihe $ \sum\limits_{n=1}^{\infty} a_n $ 
	bei $a_n \geq d_n$. \newline Dies gilt auch für $a_n \geq d_n$ erst ab einer Stelle $n_0 \in \mathbb{N}$. \\
\hline
	\begin{minipage}{100pt}
		Quotientenkrit. \\
		\fb{S.474  (7.2.2.2)}\\
		Wurzelkrit.\\
		\fb{S.474 (7.2.2.3)} 
	\end{minipage} &
	
	
	\begin{minipage}[c]{0.3\textwidth}
		\vspace{10pt}
		$ \lim\limits_{n \to \infty} \left|\dfrac{a_{n+1}}{a_n}\right| = \alpha $ der Reihe $ \sum\limits_{n=1}^{\infty}a_n \qquad$ \newline
		$\lim\limits_{n \to \infty} \sqrt[n]{\left|a_n\right|} = \alpha $ der Reihe $ \sum\limits_{n=1}^{\infty} a_n$
		
	\end{minipage}
	\begin{minipage}[c]{0.3\textwidth}
		\vspace{10pt}
		$\begin{cases}
		\alpha < 1 & $(aboslut) konvergent$\\
		\alpha = 1 & $keine Aussage!$\\
		\alpha > 1 & $divergent$
		\end{cases}$
	\end{minipage}
\\
\hline	
	
	Integralkrit. \newline
	\fb{S.475  (7.2.2.4)} &
	$
	\text{wenn}
	\left.
	\begin{cases}
		{-f(x) \text { auf dem Intervall }[1, \infty) \text { definiert }} \\ 
		{(\text { bzw. }[k, \infty))} \\ 
		{-f(x) \geq 0}\\
		{-f(x) \text { monoton fallend }}
	\end{cases}\right\}
	\Rightarrow
	\begin{cases}{\int_{1}^{\infty} f(x) d x \text { konvergent } \Leftrightarrow \text { Reihe konvergent }} \\ {\int_{1}^{\infty} f(x) d x \text { divergent } \Leftrightarrow \text { Reihe divergent }}\end{cases}
	$\\		
\hline
	Leibniz Krit. \newline
	\fb{S.476 (7.2.3.3)} &
	Die \textbf{alternierende} Reihe $ \sum\limits_{n=1}^{\infty} a_n $ ist konvergent, wenn die Folge $\langle\left|a_n\right|\rangle$ eine monoton fallende Nullfolge ($\lim\limits_{n \to \infty}
	\left|a_n\right| = 0 $) ist.\\
	&
	Monotonie mittels Verhältnis $\left( \left|\frac{a_{n+1}}{a_n}\right| \right)$, Differenz ($ |a_{n+1}| - |a_n| $) oder vollständiger Induktion beweisen.\\
	&
	Abschätzung Restglied einer alternierenden konvergenten Reihe: $|R_n|=|s-s_n|\leq|a_n+1|$\\
	\hline

\rowcolor{Gray}
\multicolumn{2}{|c|}{\textbf{Absolute und Bedinge Konvergenz} \qquad \fb{7.2.3 S.475}}\\
\hline
	Absolute Konvergenz&
	Eine Reihe $\sum\limits_{n=1}^{\infty}a_n$ heisst \textbf{absolut konvergent}, wenn die
	Reihe $\sum\limits_{n=1}^{\infty}|a_n|$ konvergent ist.\\
\hline
	Unbedingt\newline Konvergent & 
	Unbedingt Konvergent ist eine Reihe die durch umordnen einen anderen Grenzwert hat oder wird divergiert.\\
\hline
	Bedingt Konvergent &
	Unbedingt kann man umordnen, ohne dass sich konvergenz oder Grenzwert ändert.\\
\hline
\end{tabularx}

\end{center}
\end{table}	