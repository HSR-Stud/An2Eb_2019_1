% % % % % % % % % % % % % % % % % %
%DGL 2. Ordnung
% % % % % % % % % % % % % % % % % %	
% % % % % % % % % % % % % % % % % %
% neue Seite

\begin{table}[h!]
\begin{center}

\begin{tabularx}{\textwidth}{|p{120pt}|X|}
\hline
\rowcolor{Gray}
\multicolumn{2}{|c|}{\textbf{DGL 2. Ordnung}\qquad \fb{S.564}}\\
\hline
	Form  & Lösung\\
\hline
	$y''+a_1\cdot y'+a_0\cdot y=g(x)$&
	Wie bei 1. Ordnung: $Y=y_H+y_p$ \newline
	Homogene DGL: $g(x)=0$ \qquad Inhomogene DGL: $g(x)\neq 0$\\
\hline
\hline
	\rowcolor{LightCyan}
	\multicolumn{2}{|c|}{Homogene DGL $\qquad y''+a_1\cdot y'+a_0\cdot y=0$}\\
	\multicolumn{2}{|c|}{Charakt. Polynom:
	$\qquad \lambda^2+a_1\cdot\lambda+a_0=0 \qquad$ von
	$\qquad y''+a_1\cdot y'+a_0\cdot y=0$ 
	$\qquad(\lambda_{1,2} = -\frac{a_1}{2} \pm \frac{\sqrt{a_1^2 - 4a_0}}{2})$}
	\\
	\hline
	
	\multicolumn{2}{|l|}{
	$
	D = \left(\frac{a_{1}}{2}\right)^{2}-a_{0} =
	\left\{
	\begin{aligned}
		D>0: & \qquad \lambda_{1,2}=-\frac{a_{1}}{2} \pm \sqrt{\left(\frac{a_{1}}{2}\right)^{2}-a_{0}} & \qquad \in \mathbb{R} & \qquad \text{ starke D"ampfung }
		\\
		D=0: & \qquad \lambda \quad=\quad-\frac{a_{1}}{2} & \qquad \in \mathbb{R} & \qquad \text{ aperiodischer Grenzfall }
		\\
		D<0: & \qquad \lambda_{1,2}=-\frac{a_{1}}{2} \pm j \sqrt{a_{0}-\left(\frac{a_{1}}{2}\right)^{2}} & \qquad \in \mathbb{C}  \backslash \mathbb{R} & \qquad  \text{ schwache Dämpfung / Schwingfall }
	\end{aligned} \right.
	$
	}
	\\
	\hline

	\multicolumn{2}{|l|}{$(D > 0)\qquad$Falls: $\lambda_1\neq \lambda_2$ und $\lambda_{1,2} \in R\qquad$: 
	$Y_H=Ae^{\lambda_1x}+Be^{\lambda_2x\qquad}$  
	$\rbrace$ starke Dämpfung}\\
	\hline
	
	\multicolumn{2}{|l|}{$(D = 0)\qquad$
	Falls: $\lambda_1=\lambda_2$ und $\lambda_{1,2} \in R\qquad$: 
	$Y_H=e^{\lambda_1x}(A+B\cdot x)\qquad$ 
	$\rbrace$ aperiodischer Grenzfall}\\
	\hline
	
	\multicolumn{2}{|l|}{$(D < 0)\qquad$ 
	Falls $y_{H}=A \cdot \mathrm{e}^{\lambda x}=A \cdot \mathrm{e}^{-\frac{a_{1}}{2} x} \cdot \mathrm{e}^{ \pm j \sqrt{|D|} x}=A \cdot \mathrm{e}^{-\frac{a_{1}}{2} x} \cdot[\cos (\sqrt{|D|} x) \pm j \sin (\sqrt{|D|} x)]$ 
	$\rbrace$ schwache Dämpfung / Schwingfall} \\
	
	(Eigen-)Frequenz&
	$\omega=\alpha=\dfrac{\sqrt{|a_1^2 - 4a_0|}}{2} \qquad \qquad \qquad \omega=\sqrt{|D|}=\sqrt{\left|\delta^{2}-a_{0}\right|}$\\
	
	Dämpungskonstante&
	$\delta=-\frac{a_{1}}{2}$\\
	
	Resonanz&
	$\delta$ und $\omega$ stimmen überein mit Störglied\\
\hline

\hline
	\rowcolor{LightCyan}
	\multicolumn{2}{|c|}{inhomogene DGL $\qquad y''+a_1\cdot y'+a_0\cdot y=g(x)$ }\\
	Grundlöseverfahren
	\newline\newline
	(Faltungsintegral)
	&
	\begin{compactenum}
		\item Homogene DGL lösen: $g(x)=0$ setzen $\rightarrow$ ergibt $Y_H$
		\item Anfangsbedingungen in Hom. DGL einsetzen. Wenn möglich: $x_0=0\quad \newline 
				y_H(x_0)=0 \newline  y_H'(x_0)=1 \qquad$
		\item A, B bestimmen
		\item  Einsetzen der Hom. Glg. in Faltungsintegral 
		$\Rightarrow\quad y_P(x)=\int\limits_{x_o}^{x} y_H(x+x_0-t)\cdot g(t)dt$
		\item $Y=y_H+y_P$
	\end{compactenum}\\

\hline
\end{tabularx}
% % % % % % % % % % % % % % % % % %
%Ansatz in Form des Störgliedes
% % % % % % % % % % % % % % % % % %
\begin{tabularx}{\textwidth}{|p{120pt}|X|}
	\hline
	Ansatz in Form des \newline
	Störgliedes
	&
	\begin{compactenum}
		\item Homogene DGL lösen: $g(x)=0$ setzen $\rightarrow$ ergibt $Y_H$
		\item g(x) in Störgliedtabelle suchen
		\item Fall bestimmen
		\item $y_P$ aus Tabelle ablesen
		\item $Y=y_H+y_P$
	\end{compactenum}\\
\end{tabularx}

\renewcommand{\arraystretch}{1.1}
\begin{tabularx}{\textwidth}{|p{270pt}|X|}
	\hline 	$\mathbf{g(x)=p_n(x)}$ & 
		($p_n(x)$ und $q_n(x)$ sind Polynome vom gleichen Grad)\\

	 \hline	Fall 1: $a_0\neq 0$:          & $y_P = q_n(x)$\\
		Fall 2: $a_0 = 0 , a_1\neq 0$:& $y_P=x\cdot q_n(x)$\\
		Fall 3: $a_0=a_1=0$:          & $y_P=x^2\cdot q_n(x)$\\
		($a_0$ und $a_1$ beziehen sich auf die \textbf{linke Seite} der DGL) & \\
	\hline
	\hline
 		$\mathbf{g(x)=e^{bx}\cdot p_n(x)}$ & \\
	\hline	Fall 1: $b$ nicht Nullstelle des char. Polynoms:    &
		$y_P=e^{bx}\cdot q_n(x)$\\
		Fall 2: $b$ einfache Nullstelle des char. Polynoms: &
		$y_P=e^{bx}\cdot x \cdot q_n(x)$\\
		Fall 3: $b$ zweifache Nullstelle des char. Polynoms:&
		$y_P=e^{bx}\cdot x^2\cdot q_n(x)$\\
	\hline
	\hline
		$\mathbf{g(x) = e^{\alpha x}(p_n(x)\cos \beta x + q_n(x)\sin \beta x)}$ & \\
	 \hline	Fall 1: $\alpha + j\cdot\beta$ \textbf{nicht Lösung} der charakteristischen Gleichung: &
		$y_p = e^{\alpha x}\cdot(r_n(x)\cdot\cos(\beta \cdot x) + s_n(x)\cdot\sin( \beta\cdot x))$ \\
		Fall 2: $\alpha + j\cdot\beta$ \textbf{Lösung} der charakteristischen Gleichung: &
		$y_p = e^{\alpha x}\cdot \textbf{x} \cdot(r_n(x) \cdot \cos(\beta\cdot x) + s_n(x) \cdot\sin(\beta\cdot x))$\\
	\hline
\end{tabularx}
\renewcommand{\arraystretch}{2}
\end{center}
\end{table}	


\begin{table}[h!]
\begin{center}

% % % % % % % % % % % % % % % % % %
%Vorgehen bei einer DGL in From des Störgliedes
% % % % % % % % % % % % % % % % % %	
\begin{tabularx}{\textwidth}{|p{25pt}|X|}
\hline
\multicolumn{2}{|c|}{\textbf{Vorgehen bei einer DGL in Form des Störgliedes}}\\
\hline
 	1. &$Y_H$ mit $\lambda_1$ und $\lambda_2$ berechnen\\
	2. & Ordnung $n$ anhand der r.h.s der DGL bestimmen
			Koeffizient $b$ anhand der r.h.s der DGL bestimmen\newline
			(Achtung kann aus mehreren Elementen bestehen z.B. $x^2e^x + x$; Superposition)\\
		
	3. &	Anhand der Störglied Tabellen $y_p$ bestimmen\\
	4. & 	$q_n = ax^n + bx^{n-1} + \dots + cx + d$\\
	5. & 	$y_p$ ableiten und in die \textbf{ l.h.s} der DGL einsetzen. $\qquad y_p'' + a_1 y_p' + a_0y_p = f(x)$\\
	6. & 	Koeffizienten bestimmen: $\textcolor{red}{x^2e^x}\cdot 18a + xe^x(6a + 12b) + e^x(2b + 6c) = \textcolor{red}{x^2e^x}$\newline
			\begin{tabular}{ll}
				$18a = 1$ & $18a$ kommt 1mal in der r.h.s vor\\
				$(6a + 12b) = 0$ & $(6a + 12b)$ kommt 0mal vor auf der r.h.s\\
				$(2b + 6c) = 0$ & $(2b + 6c)$ kommt 0mal vor auf der r.h.s
			\end{tabular}\\
	7. & 	Koeffizienten in $y_p$ einsetzen\\
	8. & 	Wenn das Störglied $f(x)$ aus mehreren Teilen besteht (z.B. $x^2e^x + x$), Störglied auseinander nehmen und in zwei Teile $x^2e^x$ und $x$ unterteilen und Schritt 3 - 6 wiederholen\\
	9. & 	$y = Y_H + y_{p1} + y_{p2} + \dots$\\

\hline
\end{tabularx}
\begin{tabularx}{\textwidth}{|p{120pt}|X|}
Superpositionsprinzip & $f(x)=c_1f_1(x)+c_2f_2(x)$\newline
$y_1$ ist spezielle Lösung der DGL $\qquad$
$y_1''+a_1\cdot y_1'+a_0\cdot y_1=c_1f_1(x)$ \newline
$y_2$ ist spezielle Lösung der DGL $\qquad$
$y_2''+a_1\cdot y_2'+a_0\cdot y_2=c_2f_2(x)$ \newline
dann ist $y_P=c_1y_1+c_2y_2$\\
\hline
\end{tabularx}
